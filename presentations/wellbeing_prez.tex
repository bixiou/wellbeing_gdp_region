\documentclass[aspectratio=169,xcolor=dvipsnames, 11pt,mathserif]{beamer} 
\input{preamble_slides.tex}
\usepackage{multicol}
              
\begin{document}

\begin{frame}
\thispagestyle{empty}
\begin{center}
\begin{LARGE}
\textcolor{blue}{GDP per capita is a poor predictor of national well-being}
\end{LARGE}

\vspace{1cm}
\textbf{Adrien Fabre} (CNRS, CIRED)

%\vspace{-0.3cm}
%OECD/CAE \\
\medskip
\DTMlangsetup{showdayofmonth=false}
\textit{January 2024} 
%\textit{\today} 

\end{center}

\bigskip

\end{frame}

% Which is the happiest country?
% What is best predictor of national well-being: GDP pc or region?
% Literature: increasing function (although it may flatten)
% Data: WVS 111 countries w representative surveys 81-22
% Indicators of well-being

% Relation GDP - wb
% Results: variance explained by the GDP per capita is low at best
% Results: largest correlation with (log) GDP per capita is found for the average life satisfaction, the most commonly used indicator
% Results: Some indicators are not significantly related with GDP per capita, while some others (like the share of very happy people) are even decreasing with GDP per capita

% Region grouping
% Results: national well-being is better explained by the country's world region than by its GDP per capita
% Results: results hold true in each separate wave of the World Values Survey as well as in the combined dataset
% Results: set of happiest countries (with number of occurrences)

% Conclusion: 
% - small correlation with GDP, larger with region
% - richest countries are not necessarily the happiest; link is only that rich country are not sad
% - absolute income is not as determining for one's subjective well-being as is commonly thought
% - perhaps regional differences due to cultural differences in way to express feelings but feelings are same => future research: extend this to emotional well-being [if this is the case, can we use well-being indicator? we may not rule out that culture / question interpretation evolves with GDP] cf. Sacks, Stevenson, Wolfers
%     evidence against translation differences: Sandvik et al., 1993; Diener and Lucas, 1999; Scollon et al. 2005
% - probably non-material dimensions are key: dance/social bonds in LatAm vs. recession/loss of meaning after USSR => need to study mechanisms
% - poor countries can be happy too, growth not necessarily the best way to go
% - rather than regress WB on GDP, we should look for reforms that increase WB
% - would be easier to ask people what makes them happy [national well-being accounts] rather than rely on an indicator (GDP) tha barely correlates with what they want [“There should be a new rule: each and every new regulation and tax should be judged against how it will affect the UK’s GDP” Allister Heath, The Telegraph]

% Robustness checks: only_last, weight, pandemic_years, NA
% Robustness check: number of missing answers by country
% Other explanatory variables: growth, median income, etc.

\section{Introduction}

\begin{frame}{What makes a country happy?}
    \bbvsp
    \ip Which country is the happiest?
    \ip The answer is often in Scandinavia. % TODO: magazine pic
    \ip What do we mean by ``happy''? \pause Subjective well-being.
    \ee
\end{frame}

\begin{frame}{Literature}
    \bbvsp
    \ip The literature finds an increasing relationship between GDP pc and well-being.% TODO
    % Original graph is Inglehart & Klingemann (00)
    % https://ourworldindata.org/grapher/gdp-vs-happiness World Happiness Report
    % https://www.digitalinformationworld.com/2023/04/data-shows-correlation-between-gdp-per.html same source
    % https://econreview.berkeley.edu/beyond-gdp-economics-and-happiness/ same discourse as mine, The Economist chart
    % World Happiness Index and GDP per capita have correlation of .84, "same thing" for Salvatores Babones https://twitter.com/ProfBabones/status/1638924519510519809
    % various references https://www.maxroser.com/roser/notebook/happiness/
    % Great article: https://ourworldindata.org/happiness-and-life-satisfaction (already showing specificity of Latin America)
    % TODO read Inglehart et al. (08), which already points out Latin America vs. Ex-Communist
    \ip We challenge this finding.
    \ee
\end{frame}

\begin{frame}{Primer of the results}
    \bbvsp
    \ip Income is only weakly correlated with national well-being.
    \ip The relationship heavily depends on the well-being indicator chosen. \pause \\ For some indicators, the happiest country is in Africa or Latin America.\pause
    \ip Another simple variable, the country's (macro) region, is a better predictor of national well-being.
    \ee
\end{frame}

\section{Design}

\begin{frame}{Data}
    \bbvsp
    \ip World Values Survey (WVS): representative surveys on 440,000 respondents over 108 countries.
    \ip 304 country $\times$ year observations among 7 waves from 1981 to 2022.
    \ip Two subjective well-being questions:
    \bbvsp 
    \ip Happiness: ``Taking all things together, would you say you are:'' \\ ~\textit{Very happy}; \textit{Quite happy}; \textit{Not very happy}; \textit{Not at all happy}; PNR % Q46
    \ip Satisfaction: ``All things considered, how satisfied are you with your life as a whole these days?'' \textit{1-Completely dissatisfied} -- \textit{10-Completeley satisfied}; PNR % Please use this card to help with your answer.
    \ee
    \ee
    \centering \includegraphics[height=.5\textheight]{../figures/WVS_countries}
\end{frame}

\begin{frame}{What is national well-being?}
    With the two well-being questions, we can define various national indicators (all weighted using survey weights, all excluding PNR).
    \bbvsp
    \ip \textbf{Happiness (mean)}: mean happiness recoded into $-$3; $-$1; +1; +3
    \ip \textbf{Happy}: share answering \textit{Quite} or \textit{Very happy}
    \ip \textbf{Very Happy}: share answering \textit{Very happy}
    \ip \textbf{Very Unhappy}: share answering \textit{Very unhappy}
    \ip \textbf{Very Happy over Very Unhappy}: ratio of \textbf{Very Happy} over \textbf{Very Unhappy}
    \ip \textbf{Satisfaction (mean)}: mean satisfaction
    \ip \textbf{Satisfied}: share answering 6 to 10 at satisfaction
    \ip \textbf{Happy--Satisfied}: average of \textbf{Happy} and \textbf{Satisfied}\\ ~\quad This is the variable used by Inglehart \& Klingemann (2000) % https://www.staff.ncl.ac.uk/david.harvey/MKT3008/IntDev/happiness&GDP.gif SWB index by Inglehart & Klingemann (00)
    \ee 
\end{frame}
% TODO: plot happy against satisfied

\begin{frame}{How we measure income}    
    \bbvsp % As usual in this literature, 
    \ip Our preferred \textit{income} indicator is the \textbf{log} GDP per capita (pc) in \textbf{PPP} (constant 2017 \$, World Bank) \pause \\ We also use discrete indicators: \pause \bbvsp
    \ip \textbf{Income group}: quantile of income (6 quantiles)
    \ip \textbf{Income cluster} (\textit{k} = \textbf{5}, \textbf{6} or \textbf{7}): income cluster, with the \textit{k} clusters found by the \textit{k}-means algorithm
    \ee
    \ip For robustness, we also run our analyses using the log \textit{nominal} GDP pc (constant 2015 \$, World Bank) and corresponding income group and clusters.
    \ip We manually impute missing income data using IMF data.
    \bbvsp \ip For robustness, we also run our analyses without this imputation (excluding countries with missing GDP data).
    \ee
    \ee 
\end{frame}

\section{National well-being and income}

\begin{frame}{Graphical evidence}
\begin{figure}
    \only<+>{\caption{Caption} 
    \centering \includegraphics[height=.8\textheight]{../figures/WVS_countries}}
    \only<+>{\caption{Caption}  
    \centering \includegraphics[height=.8\textheight]{../figures/WVS_countries_regions}}
    \only<+>{\caption{Caption}  
    \centering \includegraphics[height=.8\textheight]{../figures/region_groupings}}
\end{figure}
\end{frame}

\begin{frame}{Share of the variance explained by GDP pc}
    % \bbvs
    % \ip For different \textit{well-being} and \textit{income} indicators, we compute the $R^2$ of the regression: $$well\text{-}being_i = \alpha + \beta income_i + u_i$$
    % \ee
    
\begin{tabular}[t]{lccccccccc}
\toprule Happiness variable & \multicolumn{2}{c}{log GDP p.c.} & \multicolumn{5}{c}{Income cluster} & & \\
  & \makecell{\,\\PPP} & \makecell{\,\\nominal} & \makecell{sextile\\PPP} & \makecell{k = 5\\PPP} & \makecell{k = 6\\PPP} & \makecell{k = 7\\PPP} & \makecell{k = 7\\nominal} & Mean & Max\\
\midrule
Very Happy & 0 & 0 & 0.04 & 0.01 & 0.01 & 0.02 & 0.04 & 0.02 & 0.04\\
Happy & 0.1 & 0.12 & 0.14 & 0.13 & 0.12 & 0.13 & 0.15 & 0.13 & 0.15\\
Very Unhappy & 0.05 & 0.06 & 0.07 & 0.07 & 0.07 & 0.07 & 0.1 & 0.07 & 0.1\\
Satisfied & 0.19 & 0.23 & 0.19 & 0.2 & 0.22 & 0.21 & 0.24 & 0.21 & 0.24\\
Satisfaction (mean) & 0.14 & 0.16 & 0.13 & 0.14 & 0.16 & 0.15 & 0.18 & 0.15 & 0.18\\
Happiness (mean) & 0.03 & 0.04 & 0.07 & 0.06 & 0.05 & 0.06 & 0.08 & 0.06 & 0.08\\
Happy + Satisfied & 0.17 & 0.21 & 0.19 & 0.19 & 0.2 & 0.2 & 0.23 & 0.2 & 0.23\\
V. Happy -- V. Unhappy & 0 & 0.01 & 0.05 & 0.02 & 0.02 & 0.03 & 0.05 & 0.03 & 0.05\\ \midrule 
Mean & 0.09 & 0.1 & 0.11 & 0.1 & 0.1 & 0.11 & 0.13 & 0.11 & 0.13\\
Max & 0.19 & 0.23 & 0.19 & 0.2 & 0.22 & 0.21 & 0.24 & 0.21 & 0.24\\ \midrule 
Number of obs. & 304 & 304 & 304 & 304 & 304 & 304 & 304 &  & \\
\bottomrule
\end{tabular}
\end{frame}

\begin{frame}{Share of the variance explained by GDP pc}
    % \bbvs
    % \ip For different \textit{well-being} and \textit{income} indicators, we compute the $R^2$ of the regression: $$well\text{-}being_i = \alpha + \beta income_i + u_i$$
    % \ee
    
\begin{tabular}[t]{lccccccccc}
\toprule & All waves & \multicolumn{6}{c}{Only selected waves} &  & \\
  & \makecell{Pop.\\weight} & \makecell{1 \& 2} & \makecell{3} & \makecell{4} & \makecell{5} & \makecell{6} & \makecell{7} & Mean & Max\\
\midrule
Very Happy & 0.05 & 0.24 & 0.07 & 0.15 & 0.07 & 0.2 & 0.25 & 0.14 & 0.25\\
Happy & 0.23 & 0.22 & 0.22 & 0.23 & 0.23 & 0.21 & 0.13 & 0.21 & 0.23\\
Very Unhappy & 0.06 & 0.22 & 0.15 & 0.16 & 0.18 & 0.12 & 0.18 & 0.15 & 0.22\\
Satisfied & 0.23 & 0.18 & 0.23 & 0.36 & 0.29 & 0.22 & 0.16 & 0.24 & 0.36\\
Satisfaction (mean) & 0.16 & 0.17 & 0.18 & 0.33 & 0.21 & 0.19 & 0.13 & 0.2 & 0.33\\
Happiness (mean) & 0.11 & 0.18 & 0.13 & 0.2 & 0.16 & 0.2 & 0.15 & 0.16 & 0.2\\
Happy + Satisfied & 0.28 & 0.21 & 0.25 & 0.32 & 0.28 & 0.24 & 0.17 & 0.25 & 0.32\\
V. Happy -- V. Unhappy & 0.06 & 0.15 & 0.08 & 0.16 & 0.09 & 0.2 & 0.21 & 0.14 & 0.21\\ \midrule 
Mean & 0.15 & 0.2 & 0.16 & 0.24 & 0.19 & 0.2 & 0.17 & 0.19 & 0.24\\
Max & 0.28 & 0.24 & 0.25 & 0.36 & 0.29 & 0.24 & 0.25 & 0.25 & 0.36\\ \midrule 
Number of obs. & 304 & 26 & 56 & 40 & 58 & 60 & 64 &  & \\
\bottomrule
\end{tabular}
\end{frame}

\begin{frame}{Share of the variance explained by GDP pc}
    % \bbvs
    % \ip For different \textit{well-being} and \textit{income} indicators, we compute the $R^2$ of the regression: $$well\text{-}being_i = \alpha + \beta income_i + u_i$$
    % \ee
    
\begin{tabular}[t]{lccccccccc}
\toprule Happiness variable & \multicolumn{2}{c}{log GDP p.c.} & \multicolumn{5}{c}{Income cluster} & & \\
  & \makecell{\,\\PPP} & \makecell{\,\\nominal} & \makecell{sextile\\PPP} & \makecell{k = 5\\PPP} & \makecell{k = 6\\PPP} & \makecell{k = 7\\PPP} & \makecell{k = 7\\nominal} & Mean & Max\\
\midrule
Very Happy & 0 & 0.01 & 0.11 & 0.03 & 0.14 & 0.07 & 0.08 & 0.06 & 0.14\\
Happy & 0.24 & 0.3 & 0.32 & 0.31 & 0.34 & 0.32 & 0.37 & 0.32 & 0.37\\
Very Unhappy & 0.24 & 0.32 & 0.35 & 0.36 & 0.37 & 0.36 & 0.48 & 0.35 & 0.48\\
Satisfied & 0.35 & 0.42 & 0.35 & 0.36 & 0.36 & 0.36 & 0.42 & 0.37 & 0.42\\
Satisfaction (mean) & 0.26 & 0.31 & 0.24 & 0.26 & 0.25 & 0.26 & 0.32 & 0.27 & 0.32\\
Happiness (mean) & 0.08 & 0.12 & 0.18 & 0.14 & 0.21 & 0.16 & 0.19 & 0.15 & 0.21\\
Happy + Satisfied & 0.32 & 0.39 & 0.34 & 0.35 & 0.35 & 0.35 & 0.41 & 0.36 & 0.41\\
V. Happy -- V. Unhappy & 0.01 & 0.03 & 0.12 & 0.05 & 0.15 & 0.09 & 0.1 & 0.08 & 0.15\\ \midrule 
Mean & 0.19 & 0.23 & 0.25 & 0.23 & 0.27 & 0.24 & 0.3 & 0.25 & 0.3\\
Max & 0.35 & 0.42 & 0.35 & 0.36 & 0.37 & 0.36 & 0.48 & 0.37 & 0.48\\ \midrule 
Number of obs. & 304 & 304 & 304 & 304 & 304 & 304 & 304 &  & \\
\bottomrule
\end{tabular}
\end{frame}

\begin{frame}{Share of the variance explained by GDP pc}
    % \bbvs
    % \ip For different \textit{well-being} and \textit{income} indicators, we compute the $R^2$ of the regression: $$well\text{-}being_i = \alpha + \beta income_i + u_i$$
    % \ee
    
\begin{tabular}[t]{lccccccccc}
\toprule Happiness variable & All waves & \multicolumn{6}{c}{Only selected waves} &  & \\
  & \makecell{Pop.\\weight} & \makecell{1 \& 2} & \makecell{3} & \makecell{4} & \makecell{5} & \makecell{6} & \makecell{7} & Mean & Max\\
\midrule
Very Happy & 0.19 & 0.3 & 0.08 & 0.36 & 0.13 & 0.37 & 0.47 & 0.27 & 0.47\\
Happy & 0.54 & 0.33 & 0.36 & 0.58 & 0.39 & 0.48 & 0.26 & 0.42 & 0.58\\
Very Unhappy & 0.25 & 0.26 & 0.28 & 0.57 & 0.44 & 0.43 & 0.34 & 0.37 & 0.57\\
Satisfied & 0.57 & 0.35 & 0.28 & 0.56 & 0.38 & 0.42 & 0.25 & 0.4 & 0.57\\
Satisfaction (mean) & 0.36 & 0.37 & 0.22 & 0.47 & 0.3 & 0.38 & 0.12 & 0.32 & 0.47\\
Happiness (mean) & 0.31 & 0.25 & 0.18 & 0.46 & 0.25 & 0.43 & 0.23 & 0.3 & 0.46\\
Happy + Satisfied & 0.57 & 0.32 & 0.32 & 0.57 & 0.39 & 0.42 & 0.24 & 0.41 & 0.57\\
V. Happy -- V. Unhappy & 0.22 & 0.22 & 0.1 & 0.38 & 0.16 & 0.41 & 0.38 & 0.27 & 0.41\\ \midrule 
Mean & 0.38 & 0.3 & 0.23 & 0.5 & 0.3 & 0.42 & 0.29 & 0.34 & 0.5\\
Max & 0.57 & 0.37 & 0.36 & 0.58 & 0.44 & 0.48 & 0.47 & 0.42 & 0.58\\ \midrule 
Number of obs. & 304 & 26 & 56 & 40 & 58 & 60 & 64 &  & \\
\bottomrule
\end{tabular}
\end{frame}

\begin{frame}{What are the happiest countries?}
    \bbvs
    \ip 
    \ip 
    \ee
\end{frame}

% Results: variance explained by the GDP per capita is low at best
% Results: largest correlation with (log) GDP per capita is found for the average life satisfaction, the most commonly used indicator
% Results: Some indicators are not significantly related with GDP per capita, while some others (like the share of very happy people) are even decreasing with GDP per capita
% Results: set of happiest countries (with number of occurrences)

\section{Region vs. GDP per capita as predictor of well-being}

\begin{frame}{Region grouping}
    \begin{figure}
        \caption{WVS countries grouped into six world regions.}  
        \centering \includegraphics[height=.8\textheight]{../figures/WVS_countries_regions} % region_groupings
    \end{figure}
\end{frame}

\begin{frame}{Region is a better predictor of national well-being than income}
    \bbvs
    \ip 
    \ip 
    \ee
\end{frame}
% Region grouping
% Results: national well-being is better explained by the country's world region than by its GDP per capita
% Results: results hold true in each separate wave of the World Values Survey as well as in the combined dataset

\section{Conclusion}
% Conclusion: 
% - small correlation with GDP, larger with region
% - richest countries are not necessarily the happiest; link is only that rich country are not sad
% - absolute income is not as determining for one's subjective well-being as is commonly thought
% - perhaps regional differences due to cultural differences in way to express feelings but feelings are same => future research: extend this to emotional well-being [if this is the case, can we use well-being indicator? we may not rule out that culture / question interpretation evolves with GDP] cf. Sacks, Stevenson, Wolfers
%     evidence against translation differences: Sandvik et al., 1993; Diener and Lucas, 1999; Scollon et al. 2005
% - probably non-material dimensions are key: dance/social bonds in LatAm vs. recession/loss of meaning after USSR => need to study mechanisms
% - poor countries can be happy too, growth not necessarily the best way to go
% - rather than regress WB on GDP, we should look for reforms that increase WB
% - would be easier to ask people what makes them happy [national well-being accounts] rather than rely on an indicator (GDP) tha barely correlates with what they want [“There should be a new rule: each and every new regulation and tax should be judged against how it will affect the UK’s GDP” Allister Heath, The Telegraph]

\end{document}